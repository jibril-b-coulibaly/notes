\documentclass[letterpaper,12pt,oneside]{report}
\usepackage[utf8]{inputenc}
\usepackage[english]{babel}
\usepackage{amssymb,amsmath}
\usepackage{multirow}
\usepackage{color}
\usepackage{graphicx}
\usepackage{bm}
\setlength{\parskip}{12pt}
\setlength{\parindent}{0pt}
\usepackage[left=.6in,right=.6in,top=0.8in,bottom=0.8in]{geometry}
%\usepackage[colorlinks=true,linkcolor=black,anchorcolor=black,citecolor=black,filecolor=black,menucolor=black,runcolor=black,urlcolor=black]{hyperref}
\usepackage{hyperref}
\usepackage{bm}


\begin{document}
\begin{center}
Personal notes on continuum mechanics
\end{center}

\chapter{Useful tools for numerical implementation}
\section{Kinematics and coordinate systems}
\subsection{Generalities}

Vectors and matrices, as linear algebra objects, should not be confused with their coordinates. The coordinates are specific to given bases, while the objects themselves are uniquely defined regardless of the basis in which they are expressed. For example, a vector $\mathbf{V}=(a,b,c)$ in $\mathbb{R}^3$ has coordinates $\mathbf{V}_{\{\mathbf{\hat{e}_1},\mathbf{\hat{e}_2},\mathbf{\hat{e}_3}\}} = (a,b,c)$ in the standard basis $\hat{B}=\{\mathbf{\hat{e}_1} =(1,0,0),\mathbf{\hat{e}_2} =(0,1,0),\mathbf{\hat{e}_3} =(0,0,1)\}$. In a basis $B=\{\mathbf{E_1} =(1/2,0,0),\mathbf{E_2} =(0,1/2,0),\mathbf{E_3} =(0,0,1/2)\}$, the coordinates of \emph{the same vector} are $\mathbf{V}_{\{\mathbf{E_1},\mathbf{E_2},\mathbf{E_3}\}} = (2a,2b,2c)$. 

Considering the coordinates $\mathbf{V}_B = (V_{E_1},V_{E_2},V_{E_3})$ of a vector $\mathbf{V}$ expressed in a basis $B=\{\mathbf{E_1},\mathbf{E_2},\mathbf{E_3}\}$ and the coordinates $\mathbf{V}_{B'} = (V_{E'_1},V_{E'_2},V_{E'_3})$ of the same vector $\mathbf{V}$ expressed in a different basis $B'=\{\mathbf{E'_1},\mathbf{E'_2},\mathbf{E'_3}\}$, we have the linear combination:
\begin{equation}
\mathbf{V} = V_{E_1}\mathbf{E_1} + V_{E_2}\mathbf{E_2} + V_{E_3}\mathbf{E_3} = V_{E'_1}\mathbf{E'_1} + V_{E'_2}\mathbf{E'_2} + V_{E'_3}\mathbf{E'_3}
\end{equation}
and the relationship between coordinates writes:
\begin{equation}
\left(\begin{array}{c}
V_{E_1} \\ V_{E_2} \\ V_{E_3}
\end{array}\right) =
\left(\begin{array}{ccc}
\frac{\partial \mathbf{E'_1}}{\partial \mathbf{E_1}} & \frac{\partial \mathbf{E'_2}}{\partial \mathbf{E_1}} & \frac{\partial \mathbf{E'_3}}{\partial \mathbf{E_1}} \\
\frac{\partial \mathbf{E'_1}}{\partial \mathbf{E_2}} & \frac{\partial \mathbf{E'_2}}{\partial \mathbf{E_2}} & \frac{\partial \mathbf{E'_3}}{\partial \mathbf{E_2}} \\
\frac{\partial \mathbf{E'_1}}{\partial \mathbf{E_3}} & \frac{\partial \mathbf{E'_2}}{\partial \mathbf{E_3}} & \frac{\partial \mathbf{E'_3}}{\partial \mathbf{E_3}} \\
\end{array}\right)
\left(\begin{array}{c}
V_{E'_1} \\ V_{E'_2} \\ V_{E'_3}
\end{array}\right)
\end{equation}

This is the classic change-of-basis relationship for vectors in linear algebra\footnote{It is important to notice, to avoid confusion, that the order of the bases is reverted with that of the terminology, i.e. in (\ref{eq:change-of-basis}) the change-of-basis matrix $\mathbf{P}_B^{B'}$ from $B$ to $B'$ multiplies the vector $\mathbf{V}_{B'}$, expressed in the destination basis $B'$, and not the vector $\mathbf{V}_B$ expressed in the source basis $B$.}: 
\begin{equation}
\mathbf{V}_B = \mathbf{P}_B^{B'} \mathbf{V}_{B'}
\label{eq:change-of-basis}
\end{equation}
where the columns of the change-of-basis matrix  $\mathbf{P}_B^{B'}$ from $B$ to $B'$ are the vectors of $B'$ expressed in the basis $B$.
%\begin{equation}
%\mathbf{E_i} = \frac{\partial \mathbf{V}}{\partial V_{E_i}}
%\end{equation}

% i.e.: $\mathbf{E_j} = {P_{\hat{B}}^B}_{ij}\mathbf{\hat{e}_i}$.
\textbf{Note:} in the following, we always assume that \emph{the bases are orthonormal}, which implies: $\mathbf{P}_{B'}^B = {\mathbf{P}_B^{B'}}^{-1} = {\mathbf{P}_B^{B'}}^{T}$.


\textbf{Note:} the only time a vector and its coordinates are expressed by the same n-tuple is when coordinates are given in the standard basis, i.e.: $\mathbf{V}=\mathbf{V}_{\hat{B}}$. 

\textbf{Note:} I am sure there is a lot of tensor algebra with co- and contra-variant tensor concepts to formalize this better, this is just a note based on my understanding and knowledge of basic linear algebra and for my practical cases of interest. 

If follows that linear algebra operations must be conducted on the vectors themselves, and not on their coordinates. This might seem trivial but has practical important in computational problems when vectors and tensors are simply given by their coordinates in a known basis, and not in their absolute (standard) coordinates.


\subsection{Deformation gradient}
The deformation gradient $\mathbf{F}$ is a mixed tensor, i.e. it maps the material fiber $\mathbf{dX}$ in the initial (reference) configuration $\Omega_0$, equipped with the basis $B_0 = \{\mathbf{E_1},\mathbf{E_2},\mathbf{E_3}\} $, to the material fiber $\mathbf{dx}$ in the current (deformed) configuration $\Omega$, equipped with the basis $B = \{\mathbf{E'_1},\mathbf{E'_2},\mathbf{E'_3}\}$:
\begin{equation}
\mathbf{dx} = \mathbf{F} \mathbf{dX}
\label{eq:F}
\end{equation}

The coordinates of $\mathbf{F}$ depend on the basis in which both the vectors $\mathbf{dx}$ and $\mathbf{dX}$ are expressed, i.e.:
\begin{equation}
\mathbf{dx}_{B} = \mathbf{F}_{B_0}^B \mathbf{dX}_{B_0}
\label{eq:F_B0_B}
\end{equation}
and using (\ref{eq:change-of-basis}), we can write the coordinates of the deformation gradient when both $\mathbf{dx}$ and $\mathbf{dX}$ are expressed in the same basis, e.g., the basis $B_0$ of the undeformed configuration $\Omega_0$:
\begin{equation}
\mathbf{dx}_{B_0} = \mathbf{F}_{B_0}^{B_0} \mathbf{dX}_{B_0} = (\mathbf{P}_{B_0}^B\mathbf{F}_{B_0}^B) \mathbf{dX}_{B_0}
\label{eq:F_B0_B0}
\end{equation}
or the basis $B$ of the deformed configuration $\Omega$:
\begin{equation}
\mathbf{dx}_B = \mathbf{F}_B^B \mathbf{dX}_B = (\mathbf{F}_{B_0}^B \mathbf{P}_{B_0}^B) \mathbf{dX}_B
\label{eq:F_B_B}
\end{equation}
This is the classic change-of-basis relationship for matrices in linear algebra:
\begin{equation}
\mathbf{F}_B^B  = {\mathbf{P}_{B_0}^B}^T \mathbf{F}_{B_0}^{B_0} \mathbf{P}_{B_0}^B
\label{eq:change_basis_matrix}
\end{equation}
%with $\mathbf{P_{B_0}^B} = \mathbf{P_B^{B_0}}^{-1}$. The columns of the change-of-basis matrix $\mathbf{P_{B_0}^B}$ from basis $B_0$ to basis $B$ are the vectors of $B$ expressed in the basis $B_0$, i.e.: $\mathbf{e_j} = \left(P_{B_0}^B\right)_{ij}\mathbf{E_i}$. Conversely, the columns of the change-of-basis matrix $\mathbf{P_B^{B_0}}$ from basis $B$ to basis $B_0$ are the vectors of $B_0$ expressed in the basis $B$, i.e.: $\mathbf{E_j} = \left(P_B^{B_0}\right)_{ij}\mathbf{e_i}$. Additionally, if we assume (always the case in mechanics?) that both $B_0$ and $B$ are orthonormal bases and that the transformation from $B_0$ to $B$ is a rotation, the inverse of the change-of-basis matrix is its transpose, i.e.: $\mathbf{P_{B_0}^B}^{-1} = \mathbf{P_{B_0}^B}^{T}$, that gives:
%\begin{equation}
%\left(P_{B_0}^B\right)_{ij} = \mathbf{E_i} \cdot \mathbf{e_j}
%\label{eq:coordinates_P_B0_B}
%\end{equation}
%\begin{equation}
%\left(P_B^{B_0}\right)_{ij} = \mathbf{e_i} \cdot \mathbf{E_j}
%\label{eq:coordinates_P_B_B0}
%\end{equation}

%\subsection{Displacement gradient}


\subsection{Strain tensor}

Polar decomposition can be applied to the deformation gradient: $\mathbf{F} = \mathbf{RU}$, defining the right Cauchy-Green tensor $\mathbf{C} = \mathbf{U}^2 = \mathbf{F}^T\mathbf{F}$. This tensor is a Lagrangian measure of deformation and is defined on $\Omega_0$. From there, the Green-Lagrange strain tensor $\mathbf{E}$ is defined as:
\begin{equation}
\mathbf{E} = \frac{1}{2}\left(\mathbf{F}^T \mathbf{F} - \mathbf{1}\right)
\label{eq:E}
\end{equation}
where $\mathbf{1}$ represents the identity.

The Green-Lagrange strain tensor characterizes the deformation. It is not a mixed tensor and because it is solely defined on $\Omega_0$, its coordinates are independent of the basis in which the deformed configuration is expressed. Indeed from (\ref{eq:F_B0_B0}) we observe that for any basis $B'$ of $\Omega$:
\begin{equation}
{\mathbf{F}_{B_0}^{B_0}}^T \mathbf{F}_{B_0}^{B_0} = (\mathbf{P}_{B_0}^{B'} \mathbf{F}_{B_0}^{B'})^T \mathbf{P}_{B_0}^{B'} \mathbf{F}_{B_0}^{B'} = {\mathbf{F}_{B_0}^{B'}}^T {\mathbf{P}_{B_0}^{B'}}^T  \mathbf{P}_{B_0}^{B'} \mathbf{F}_{B_0}^{B'} = {\mathbf{F}_{B_0}^{B'}}^T \mathbf{F}_{B_0}^{B'}
\end{equation}
so that in terms of coordinates, and for any basis $B'$ of $\Omega$, we have:
\begin{equation}
\mathbf{E}_{B_0} = \frac{1}{2}\left({\mathbf{F}_{B_0}^{B'}}^T \mathbf{F}_{B_0}^{B'} - \mathbf{I}\right)
\label{eq:E_coordinates}
\end{equation}
with $\mathbf{I} = \mathrm{diag}(1,1,1)$ the coordinates of the identity when the source and destination bases are the same ($B_0$ in $\Omega_0$ here). The coordinates of the strain tensor obviously depend on the choice of a basis $B_0$ for the reference configuration $\Omega_0$, this classically highlights principal strains etc. From (\ref{eq:change_basis_matrix}), we can write:
\begin{equation}
\mathbf{E}_B = {\mathbf{P}_{B_0}^B}^T \mathbf{E}_{B_0} \mathbf{P}_{B_0}^B
\label{eq:E_B}
\end{equation}

\paragraph{Infinitesimal strain} In some applications where strains are small, the Green-Lagrange strain is linearized into the infinitesimal strain tensor:
\begin{equation}
\boldsymbol{\varepsilon} = \frac{1}{2}\left(\mathbf{F}^T + \mathbf{F}\right) - \mathbf{1}
\label{eq:E_infinitesimal}
\end{equation}
The infinitesimal strain tensor should also be independent of the basis in which the deformed configuration is expressed. However, (\ref{eq:E_infinitesimal}) clearly shows coordinates $\mathbf{F}_{B_0}^{B'}$ cannot be used directly here, unlike for $\mathbf{E}$.

Notice that the infinitesimal strain is obtained as a first-order Taylor series expansion of Green-Lagrange strain at $\mathbf{F}=\mathbf{1}$, i.e.:
\begin{equation}
\mathbf{E} = \frac{1}{2}\left[(\mathbf{F}-\mathbf{1})^T \mathbf{1} + \mathbf{1}^T(\mathbf{F}-\mathbf{1})\right] + o(\mathbf{F}^2)
\label{eq:E_Taylor}
\end{equation}

The coordinates of the identity $\mathbf{1}$ from the basis $B_0$ of $\Omega_0$ to the basis $B$ of $\Omega$,  are not $\mathbf{I} = \mathrm{diag}(1,1,1)$. Indeed, from (\ref{eq:change-of-basis}), we can see that the change-of-basis matrix $\mathbf{P}_{B_0}^B$ can also be defined as the coordinates of the identity application from the basis $B$ of $\Omega$ to the basis $B_0$ $\Omega_0$, i.e.:
\begin{equation}
\mathbf{1}_{B}^{B_0} = \mathbf{P}_{B_0}^{B}
\end{equation}
\begin{equation}
\mathbf{1}_{B_0}^B = {\mathbf{P}_{B_0}^{B}}^T
\end{equation}

For any basis $B'$ of $\Omega$, (\ref{eq:E_Taylor}) can be expressed in terms of the coordinates as:
\begin{equation}
\boldsymbol{\varepsilon}_{B_0} = \frac{1}{2}\left[(\mathbf{F}_{B_0}^{B'}-\mathbf{1}_{B_0}^{B'})^T \mathbf{1}_{B_0}^{B'} + {\mathbf{1}_{B_0}^{B'}}^T(\mathbf{F}_{B_0}^{B'}-\mathbf{1}_{B_0}^{B'})\right]
\label{eq:epsilon_coordinates}
\end{equation}
and the infinitesimal strain (\ref{eq:E_infinitesimal}) writes:
\begin{equation}
\boldsymbol{\varepsilon}_{B_0} = \frac{1}{2}\left({\mathbf{F}_{B_0}^{B'}}^T {\mathbf{P}_{B_0}^{B'}}^T + \mathbf{P}_{B_0}^{B'} \mathbf{F}_{B_0}^{B'}\right) - \mathbf{I}
\end{equation}

\subsection{Example and application}

FEM software allow implementation of user-defined materials. In COMSOL Multiphysics, there is a stress--deformation socket that allows the implementation of constitutive models with the deformation gradient as an input and the second Piola-Kirchhoff stress as an output.

The deformation gradient is expressed between the global coordinate system of the deformed configuration $\mathbf{dx}_B$ and the local coordinate system of the material $\mathbf{dx}_{B_0}$. The numerical 3-by-3 matrix input variable \texttt{F[3][3]} for the deformation gradient in the code is therefore expressed in its \emph{coordinates} $\mathbf{F}_{B_0}^B$, with $B_0$ and $B$ the respective bases of the initial and deformed configurations $\Omega$ and $\Omega_0$.

\paragraph{Green-Lagrange strain tensor} As shown in (\ref{eq:E_coordinates}), the Green-Lagrange strain tensor $\mathbf{E}$ is independent of the basis $B$ of the deformed configuration $\Omega$. That means the following algorithm (\texttt{C} pseudo-code) to compute the Green-Lagrange strain tensor would provide the correct result:
\begin{verbatim}
// Correct computation of Green-Lagrange strain E
E[3][3] = {{0,0,0},{0,0,0},{0,0,0}};
delta[3][3] = {{1,0,0},{0,1,0},{0,0,1}};
for (i=0 ; i<3 ; i++)
  for (j=0 ; j<3 ; j++)
    for (k=0 ; k<3 ; k++)
      E[i][j] += 0.5*(F[k][i]*F[k][j] - delta[i][j])
\end{verbatim}

\paragraph{Infinitesimal strain tensor}
However, the infinitesimal strain tensor $\mathbf{\varepsilon}$ cannot be calculated using the input coordinates \texttt{F[3][3]}, along with the formula (\ref{eq:E_infinitesimal}), because that formula applies to linear algebra vectors and not their coordinates. If our constitutive model is formulated in terms of infinitesimal/engineering strain, the following naive implementation would return a \emph{wrong result} for the infinitesimal strain!
\begin{verbatim}
// Wrong computation of infinitesimal strain \varepsilon
eps[3][3] = {{0,0,0},{0,0,0},{0,0,0}};
delta[3][3] = {{1,0,0},{0,1,0},{0,0,1}};
for (i=0 ; i<3 ; i++)
  for (j=0 ; j<3 ; j++)
    eps[i][j] = 0.5*(F[i][j] + F[j][i]) - delta[i][j]
\end{verbatim}





















%The coordinates matrix $\mathbf{T}$ of the basis vectors of $B$ expressed as a function of the basis vectors of $B_0$
%\begin{equation}
%\mathbf{T} =
%\begin{array}{cc}
%& \begin{array}{ccc} X & Y & Z \end{array} \\
%\begin{array}{c} x \\ y \\ z \end{array} &
%\left(\begin{array}{ccc}
%T_{11} & T_{12} & T_{13} \\
%T_{21} & T_{22} & T_{23} \\
%T_{31} & T_{32} & T_{33}
%\end{array}\right)
%\end{array}
%\end{equation}



\section{Derivative with respect to deformation gradient}

This derivative comes up when computing the Jacobian of stress in function of the deformation gradient (e.g., general stress-deformation socket of external materials in COMSOL). We have:
\begin{equation}
\frac{\partial \mathbf{S}}{\partial \mathbf{F}} = \frac{\partial \mathbf{S}}{\partial \mathbf{E}} \frac{\partial \mathbf{E}}{\partial \mathbf{F}}
\end{equation}
where the first derivative $\frac{\partial \mathbf{S}}{\partial \mathbf{E}}$ is simply the stiffness tensor of the material.

The second derivative $\frac{\partial \mathbf{E}}{\partial \mathbf{F}}$ expresses the variations in strain tensor as a function of the variations in deformation gradient:
\begin{equation}
d\mathbf{E} = \frac{\partial \mathbf{E}}{\partial \mathbf{F}} d\mathbf{F}
\end{equation}

We use Voigt notation for the strain tensor (symmetric) and row-major order expansion for the deformation gradient (non-symmetric) in order to express the tensor as 2D matrices:
\begin{equation}
\mathbf{E}_{Voigt} = (E_{11},E_{22},E_{33},E_{23},E_{13},E_{12})^T
\end{equation}
\begin{equation}
\mathbf{F}_{row-major} = (F_{11},F_{12},F_{13},F_{21},F_{22},F_{23},F_{31},F_{32},F_{33})^T
\end{equation}

\subsection{Green-Lagrange strain tensor}
The Green-Lagrange strain tensor is defined by:
\begin{equation}
\mathbf{E} = \frac{1}{2}\left(\mathbf{F}^T \mathbf{F} - \mathbf{1}\right)
\end{equation}
In index notation:
\begin{equation}
E_{ij} = \frac{1}{2}\left(F_{ki} F_{kj} - \delta_{ij}\right)
\end{equation}
from which we can write:
\begin{equation}
\frac{\partial E_{ij}}{\partial F_{kl}} = \frac{1}{2}\left(F_{kj} \delta_{li} + F_{ki} \delta_{lj}\right)
\end{equation}
and gives the following 6-by-9 matrix, verifying $d\mathbf{E}_{Voigt} = \frac{\partial \mathbf{E}}{\partial \mathbf{F}} d\mathbf{F}_{row-major}$:
\begin{equation}
\frac{\partial \mathbf{E}}{\partial \mathbf{F}} = \frac{1}{2}
\left[\begin{array}{ccccccccc}
2F_{11} & 0 & 0 & 2F_{21} & 0 & 0 & 2F_{31} & 0 & 0 \\
0 & 2F_{12} & 0 & 0 & 2F_{22} & 0 & 0 & 2F_{32} & 0 \\
0 & 0 & 2F_{13} & 0 & 0 & 2F_{23} & 0 & 0 & 2F_{33} \\
0 & F_{13} & F_{12} & 0 & F_{23} & F_{22} & 0 & F_{33} & F_{32} \\
F_{13} & 0 & F_{11} & F_{23} & 0 & F_{21} & F_{33} & 0 & F_{31} \\
F_{12} & F_{11} & 0 & F_{22} & F_{21} & 0 & F_{32} & F_{31} & 0
\end{array}\right]
\end{equation}

\subsection{Linearized (engineering) strain tensor}
The linearized strain tensor is defined by:
\begin{equation}
\boldsymbol{\varepsilon} = \frac{1}{2}\left(\mathbf{F}^T + \mathbf{F}\right) - \mathbf{1}
\end{equation}
In index notation:
\begin{equation}
\varepsilon_{ij} = \frac{1}{2}\left(F_{ji} + F_{ij}\right) - \delta_{ij}
\end{equation}
from which we can write:
\begin{equation}
\frac{\partial \varepsilon_{ij}}{\partial F_{kl}} = \frac{1}{2}\left(\delta_{jk}\delta_{il} + \delta_{ik}\delta_{jl} \right)
\end{equation}
and gives the following 6-by-9 matrix, verifying:
\begin{equation}
\frac{\partial \boldsymbol{\varepsilon}}{\partial \mathbf{F}} = \frac{1}{2}
\left[\begin{array}{ccccccccc}
2 & 0 & 0 & 0 & 0 & 0 & 0 & 0 & 0 \\
0 & 0 & 0 & 0 & 2 & 0 & 0 & 0 & 0 \\
0 & 0 & 0 & 0 & 0 & 0 & 0 & 0 & 2 \\
0 & 0 & 0 & 0 & 0 & 1 & 0 & 1 & 0 \\
0 & 0 & 1 & 0 & 0 & 0 & 1 & 0 & 0 \\
0 & 1 & 0 & 1 & 0 & 0 & 0 & 0 & 0
\end{array}\right]
\end{equation}
We can see that the small strain approximation corresponds to a deformation gradient equal to the identity, i.e. undeformed: $\frac{\partial \boldsymbol{\varepsilon}}{\partial \mathbf{F}} = \frac{\partial \mathbf{E}}{\partial \mathbf{F}}(\mathbf{F}=\mathbf{1})$.

\subsection{Invariants of the deformation gradient}

\subsubsection{First invariant}

First invariant:
\begin{equation}
I_1 = \mathrm{tr}(\bm{F}^T\bm{F})
\end{equation}
In index notation:
\begin{equation}
I_1 = F_{ij} F_{ij}
\end{equation}
from which we can write:
\begin{equation}
\frac{\partial (F_{ij} F_{ij})}{\partial F_{kl}} = 2 F_{kl}
\end{equation}
Finally:
\begin{equation}
\frac{\partial I_1}{\partial \bm{F}} = \frac{ \partial \ \mathrm{tr}(\bm{F}^T\bm{F})}{\partial \bm{F}} = 2 \bm{F}
\end{equation}


\section{Volumetric and deviatoric decomposition of stress and strain}

\subsection{Stress}
Let the stress tensor expressed as:
\begin{equation}
\boldsymbol{\sigma} = 
\left[\begin{array}{ccc}
\sigma_{11} & \sigma_{12} & \sigma_{13} \\
\sigma_{12} & \sigma_{22} & \sigma_{23} \\
\sigma_{13} & \sigma_{23} & \sigma_{33}
\end{array}\right]
\end{equation}
The mean stress is defined as: $p = \mathrm{tr}(\boldsymbol{\sigma})/3$, and the deviatoric stress as: $\mathbf{s} = \boldsymbol{\sigma} - p \mathbf{1}$. In Voigt notation, we can therefore write:
\begin{equation}
\boldsymbol{\sigma} = 
\left(\begin{array}{c}
\sigma_{11} \\ \sigma_{22} \\ \sigma_{33} \\ \sigma_{23} \\ \sigma_{13} \\ \sigma_{12}
\end{array}\right)
 = 
\left(\begin{array}{c}
s_{11} \\ s_{22} \\ s_{33} \\ s_{23} \\ s_{13} \\ s_{12}
\end{array}\right)
+ p
\left[\begin{array}{cccccc}
1 & 0 & 0 & 0 & 0 & 0 \\
0 & 1 & 0 & 0 & 0 & 0 \\
0 & 0 & 1 & 0 & 0 & 0 \\
0 & 0 & 0 & 0 & 0 & 0 \\
0 & 0 & 0 & 0 & 0 & 0 \\
0 & 0 & 0 & 0 & 0 & 0
\end{array}\right]
\end{equation}


\subsection{Strain}
Let the strain tensor expressed as:
\begin{equation}
\mathbf{E} = 
\left[\begin{array}{ccc}
E_{11} & E_{12} & E_{13} \\
E_{12} & E_{22} & E_{23} \\
E_{13} & E_{23} & E_{33}
\end{array}\right]
\end{equation}
The volumetric strain is defined as: $E_v = \mathrm{tr}(\mathbf{E})$, and the deviatoric strain as: $\mathbf{E}_s = \mathbf{E} - E_v \mathbf{1} /3$. In Voigt notation, we can therefore write:
\begin{equation}
\mathbf{E} = 
\left(\begin{array}{c}
E_{11} \\ E_{22} \\ E_{33} \\ E_{23} \\ E_{13} \\ E_{12}
\end{array}\right)
 = 
\left(\begin{array}{c}
{E_s}_{11} \\ {E_s}_{22} \\ {E_s}_{33} \\ {E_s}_{23} \\ {E_s}_{13} \\ {E_s}_{12}
\end{array}\right)
+ E_v
\left[\begin{array}{cccccc}
1/3 & 0 & 0 & 0 & 0 & 0 \\
0 & 1/3 & 0 & 0 & 0 & 0 \\
0 & 0 & 1/3 & 0 & 0 & 0 \\
0 & 0 & 0 & 0 & 0 & 0 \\
0 & 0 & 0 & 0 & 0 & 0 \\
0 & 0 & 0 & 0 & 0 & 0
\end{array}\right]
\end{equation}



\chapter{Concepts}

\href{https://fr.wikipedia.org/wiki/Tenseur}{French Wikipedia on tensors}
A second-order tensor is a bilinear form. That clarifies a lot of things for me with regards to the space strain tensor is defined in etc. Additionally its change of basis is therefore not $B=P^{-1}AP$, but $B=P^TAP$. This is identical when bases are orthonormal but if they are not, loosely speaking we can see that it scales $A$ by a factor $P^2$. This shows the definition of strain as giving the relative squared stretch of a fiber.

\href{https://fr.wikipedia.org/wiki/Matrice_de_passage}{French Wikipedia on change-of-basis}


The velocity gradient $\bm{L} = \dot{\bm{F}}\bm{F}^{-1}$ defines the rate of deformation: $\bm{D} = (\bm{L}^T + \bm{L})/2$.
In the absence of rotation ($\bm{F} = \bm{RU}$, $\bm{R} = \bm{I}$), the rate of deformation  is equal to the time derivative of the logarithmic strain: $\bm{D} = \partial \ln(\bm{U})/\partial t$.


\chapter{Plasticity}

The second invariant of the deviatoric stress $J_2 = 1/2 tr(S^2)$ so that the radius of the yield surface in the pi-plane is not $\sqrt{J_2}$ but $\sqrt{2J_2}$.

\end{document}

