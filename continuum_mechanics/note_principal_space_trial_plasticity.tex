\documentclass[letterpaper,12pt]{article}
\usepackage[utf8]{inputenc}
\usepackage[english]{babel}
\usepackage[left=1in,right=1in]{geometry}
\usepackage{amsmath}
\usepackage{amssymb}
\usepackage{float}
\usepackage{url}
\usepackage[colorlinks=true,linkcolor=black,anchorcolor=black,citecolor=black,filecolor=lack,menucolor=black,runcolor=black,urlcolor=black]{hyperref}

\renewcommand{\thesection}{\arabic{section}} % Do not number chapters

\setlength{\parskip}{12pt}

\begin{document}

\bibliographystyle{apalike}
\renewcommand\bibname{References}

\begin{center}
{\large
Note on the principal direction of the trial state and final state in return mapping algorithm for isotropic elastoplasticity}

Jibril B. Coulibaly (\href{mailto:jbcouli@sandia.gov}{jbcouli@sandia.gov})
\end{center}


\section{Return mapping}

Return mapping is a prediction-correction algorithm to determine the stress state of an elastoplastic material under strain-controleld loading. A strain increment $\Delta \boldsymbol{\varepsilon}$ is applied between time $t$, when all quantities are known, and $t+\Delta t$, where all quantities are sought.

\subsection{Prediction - trial state}

A trial state is obtained by first predicting the stress assuming the strain increment is elastic: $\Delta\boldsymbol{\varepsilon}=\Delta\boldsymbol{\varepsilon}^e$, giving:
\begin{equation}
\boldsymbol{\sigma}^{tr} = \boldsymbol{\sigma}_t + \boldsymbol{C}:\Delta \boldsymbol{\varepsilon}
\end{equation}
with $\boldsymbol{C}$ the elastic tangent operator defined by: $\boldsymbol{\sigma} = \mathbf{C} : \boldsymbol{\varepsilon}^e$.

%\begin{itemize}
%\item if the trial stress state is inside the elastic domain, i.e., yield function is negative $F\leq0$, the stress state is admissible
%\item if the trial stress state is outside the elastic domain, i.e., yield function is positive $F>0$, the stress state is not admissible and must be "returned" onto the yield surface $F=0$ through plastic correction
%\end{itemize}


\subsection{Correction}

For both rate-independent and rate-dependent plasticity (albeit different formulations), the rate of plastic strain can be expressed as:
\begin{equation}
\dot{\boldsymbol{\varepsilon}^p} = \varphi\frac{\partial G}{\partial \boldsymbol{\sigma}}
\end{equation}
where $\varphi$ is a plastic multiplier, characterizing the intensity of the plastic flow, and $G$ is the plastic potential. Both $\varphi$ and $G$ are scalar functions of the stress invariants.

The elastic strain increment is given by: $\Delta \boldsymbol{\varepsilon}^e = \Delta \boldsymbol{\varepsilon} - \Delta \boldsymbol{\varepsilon}^p$, so that the stress correction from the trial state is:
\begin{equation}
\boldsymbol{\sigma}_{t+\Delta t} = \boldsymbol{\sigma}^{tr} - \boldsymbol{C}:\Delta \boldsymbol{\varepsilon}^p_{t+\Delta t}
\label{eq:corrected}
\end{equation}
which is most often a non-linear equation solved for $\boldsymbol{\varepsilon}^p_{t+\Delta t}$ using an iterative process.

\section{Principal directions and principal stress}

\subsection{Generalities}

Any symmetric real matrix $\mathbf{A}$ (e.g., 3x3) can be diagonalized as $\mathbf{A} = \mathbf{P}\mathbf{D}\mathbf{P}^T$ into a principal space given by a set or orthonormal eigenvectors $\mathbf{N}_1$, $\mathbf{N}_2$, $\mathbf{N}_3$, forming an orthogonal matrix $\mathbf{P} = [\mathbf{N}_1 \mathbf{N}_2 \mathbf{N}_3]$, i.e., $\mathbf{P}^{-1} = \mathbf{P}^T$. The corresponding eigenvalues are given by $\mathbf{D} = \mathrm{diag}[A_1, A_2, A_3]$, which allows the spectral decomposition:
\begin{equation}
\mathbf{A} = A_1 \underbrace{\mathbf{N}_1 \otimes \mathbf{N}_1}_{\mathbf{M}^1} + A_2 \underbrace{\mathbf{N}_2 \otimes \mathbf{N}_2}_{\mathbf{M}^2} + A_3 \underbrace{\mathbf{N}_3 \otimes \mathbf{N}_3}_{\mathbf{M}^3}
\label{eq:spectral}
\end{equation}
where the $\mathbf{M}_i$ are orthogonal and sum up to the identity matrix.
\subsection{Application to stress and plastic potential}

Stress is represented by real symmetric matrices who share the same principal directions, so that:
\begin{equation}
\boldsymbol{\sigma}  = \sigma_1 \mathbf{M}^1 + \sigma_2 \mathbf{M}^2 + \sigma_3 \mathbf{M}^3 = \sum_{i=1}^{3} \sigma_i \mathbf{M}^i
\end{equation}

The invariants of the stress tensor are expressed as a function of the principal stresses $\sigma_i$, so that rotation of the principal directions $\mathbf{M}_i$ has no influence on the value of the invariants. As a consequence the plastic potential function $G$ is independent of the principal stress direction, i.e.:
\begin{equation}
\frac{\partial G}{\partial \boldsymbol{\sigma}} = \sum_{i=1}^3 \frac{\partial G}{\partial \sigma_i} \mathbf{M}^i + \underbrace{\frac{\partial G}{\partial \mathbf{M}^i}}_{=0} \sigma_i
\end{equation}

\subsection{Principal direction of trial stress and final stress}
These considerations are valid for any stress state, at any time. In particular, when applied at the unknown converged state at time $t+\Delta t$, equation (\ref{eq:corrected}) becomes:
\begin{equation}
\sum_{i=1}^{3} \sigma_i \mathbf{M}^i = \boldsymbol{\sigma}^{tr} - \boldsymbol{C}:\sum_{i=1}^3 \varphi\frac{\partial G}{\partial \sigma_i} \mathbf{M}^i
\end{equation}
which can be rearranged as:
\begin{equation}
\boldsymbol{\sigma}^{tr} = \sum_{i=1}^{3} \sigma_i \mathbf{M}^i + \boldsymbol{C}:\sum_{i=1}^3 \varphi\frac{\partial G}{\partial \sigma_i} \mathbf{M}^i
\end{equation}
We can see that the trial stress, which is known, can be expressed as a spectral decomposition (\ref{eq:spectral}), with the same basis matrices $\mathbf{M}^i$ that the stress at time $t+\Delta t$.

\paragraph{Conclusion:} this indicates that, for isotropic elastoplasticity, the principal directions of stress at time $t+\Delta t$, $\boldsymbol{\sigma}_{t+\Delta t}$, which is unknown, are the same as the principal directions of the trial stress, $\boldsymbol{\sigma}^{tr}$, which is known. This is a very strong and nice result!

\end{document}